\chapter{FAQ}

M : Kalo Intership II atau TA harus buat aplikasi ?
D : Ga harus buat aplikasi tapi harus ngoding

M : Pa saya bingung mau ngapain, saya juga bingung mau presentasi apa?
D : Makanya baca de, buka jurnal topik `ganteng' nah kamu baca dulu sehari 5 kali ya, 4 hari udah 20 tuh. Bingung itu tanda kurang wawasan alias kurang baca.

M : Pa saya sudah cari jurnal terindeks scopus tapi ga nemu.
D : Kamu punya mata de? coba dicolok dulu. Kamu udah lakuin apa aja? tolong di list laporkan ke grup Tingkat Akhir. Tinggal buka google scholar klik dari tahun 2014, cek nama jurnalnya di scimagojr.com beres.

M : Pa saya belum dapat tempat intership, jadi ga tau mau presentasi apa?
D : kamu kok ga nyambung, yang dipresentasikan itu yang kamu baca bukan yang akan kamu lakukan.

M : Pa ini jurnal harus yang terindex scopus ga bisa yang lain ?
D : Index scopus menandakan artikel tersebut dalam standar semantik yang mudah dipahami dan dibaca serta bukan artikel asal jadi. Jika diluar scopus biasanya lebih sukar untuk dibaca dan dipahami karena tidak adanya proses review yang baik dan benar terhadap artikel.

M : Pa saya tidak mengerti
D : Coba lihat standar alasan

M : Pa saya bingung
D : Coba lihat standar alasan

M : Pa saya sibuk
D : Mbahmu....

M : Pa saya ganteng
D : Ndasmu....

M : Pa saya kece
D : wes karepmu lah....


Biasanya anda memiliki alasan tertentu jika menghadapi kendala saat proses bimbingan, disini saya akan melakukan standar alasan agar persepsi yang diterima sama dan tidak salah kaprah. Penggunaan kata alasan tersebut antara lain :

1. Tidak Mengerti : anda boleh menggunakan alasan ini jika anda sudah melakukan tahapan membaca dan meresumekan 15 jurnal. Sudah mencoba dan mempraktekkan teorinya dengan mencari di youtube dan google minimal 6 jam sehari selama 3 hari berturut-turut.

2. Bingung : anda boleh mengatakan alasan bingung setelah maksimal dalam berusaha menyelesaikan tugas bimbingan dari dosen(sudah dilakukan semua). Anda belum bisa mengatakan alasan bingung jika anda masih belum menyelesaikan tugas bimbingan dan poin nomor 1 diatas. Setelah anda menyelesaikan tugas bimbingan secara maksimal dan tahap 1 poin diatas, tapi anda masih tetap bingung maka anda boleh memakai alasan ini.