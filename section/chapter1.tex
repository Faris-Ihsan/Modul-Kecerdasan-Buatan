\chapter{Mengenal Kecerdasan Buatan dan Scikit-Learn}
Buku umum teori lengkap yang digunakan memiliki judul\textit{Artificial intelligence: a modern approach}\cite{russell2016artificial}.  
Untuk pratikum sebelum UTS menggunakan buku \textit{Python Artificial Intelligence Projects for Beginners}\cite{eckroth2018python}. Buku pelengkap penunjang penggunaan python menggunakan buku \textit{Python code for Artificial Intelligence: Foundations of Computational Agents}\cite{poole2017python}.
Dengan praktek menggunakan python 3 dan editor anaconda dan library python scikit-learn.
Tujuan pembelajaran pada pertemuan pertama antara lain:
\begin{enumerate}
\item
Mengerti definisi kecerdasan buatan, sejarah kecerdasan buatan, perkembangan dan penggunaan di perusahaan
\item
Memahami cara instalasi dan pemakaian sci-kit learn
\item
Memahami cara penggunaan variabel explorer di spyder
\end{enumerate}
Tugas dengan cara dikumpulkan dengan pull request ke github dengan menggunakan latex pada repo yang dibuat oleh asisten riset.

\section{Teori}
Praktek teori penunjang yang dikerjakan :
\begin{enumerate}
\item
Definisi, Sejarah dan perkembangan Kecerdasan Buatan.
\begin{itemize}
\item Definisi 
\par \setlength{\parindent}{10ex}
AI (Artificial Intelligence) atau dikenal dengan nama Kecerdasan Buatan merupakan implementasi kecerdasan yang dimiliki manusia pada mesin atau sebuah program sehingga program tersebut dapat berpikir selayaknya seorang manusia. 
\par Pengetahuan yang digunakan oleh AI merupakan pengetahuan yang berbentuk data yang diinputkan oleh manusia pembuatnya. 	Pada AI terdapat poin penting yang berupa \emph{learning, reasoning,} dan \emph{self correction}. Pada tahap learning, AI dapat 		mengimprovisasi pengetahuannya tanpa bantuan manusia. AI melakukan improvisasi dengan menggunakan data seadanya yang pernah diinputkan oleh manusia. Reasoning merupakan penalaran atau pemberian alasan atau pemecahan oleh AI dengan menggunakan pengetahuan yang dimilikinya. Selain itu, self correction adalah kemampuan untuk memperbaiki keputusan yang salah ambil dan perupakan proses belajar AI melalui pengamatan sekitarnya.
\par AI merupakan kecerdasan buatan yang didalamnya terdapat faktor \emph{Acting Humanly} (Tindakan AI yang selayaknya manusia), \emph{Thinking Humanly} (Pola pikir selayaknya manusia), \emph{Think Rationally }(Berpikir Rasional selayaknya manusia), dan \emph{Act Rationally} (Bertindak rasional selayaknya manusia).

	\item Sejarah dan Perkembangan
\end{itemize}


\item
Definisi supervised learning, klasifikasi, regresi dan unsupervised learning. Data set, training set dan testing set.
\begin{itemize}

\item Supervised Learning dan Unsupervised Learning
\par \setlength{\parindent}{10ex}
Supervised Learning merupakan Algoritma yang digunakan dalam sebuah data science. Supervised learning digunakan pada algoritma kecerdasan buatan dan digunakan untuk membuat prediksi atau klasifikasi. Supervised learning berarti kita memasukan sebuah pengetahuan baru kedalam otak AI.
\par Sedangkan pada \emph{Unsupervised Learning} berarti kita tidak perlu memasukan pengetahuan yang baru ke dalam otak AI. Hal tersebut karena pada algoritma \emph{unsupervised learning} ini, AI akan improvisasi tanpa harus dilatih terlebih dahulu. 

\item Klasifikasi dan Regresi
\item Dataset, Training Set dan Testing Set
\end{itemize}
\end{enumerate}

\section{Instalasi}
Membuka https://scikit-learn.org/stable/tutorial/basic/tutorial.html. Dengan menggunakan bahasa yang mudah dimengerti dan bebas plagiat. 
Dan wajib skrinsut dari komputer sendiri.
\begin{enumerate}
\item
Instalasi library scikit dari anaconda, mencoba kompilasi dan uji coba ambil contoh kode dan lihat variabel explorer[hari ke 1](10)
\item
Mencoba Loading an example dataset, menjelaskan maksud dari tulisan tersebut dan mengartikan per baris[hari ke 1](10)
\item
Mencoba Learning and predicting, menjelaskan maksud dari tulisan tersebut dan mengartikan per baris[hari ke 2](10)
\item
mencoba Model persistence, menjelaskan maksud dari tulisan tersebut dan mengartikan per baris[hari ke 2](10)
\item 
Mencoba Conventions, menjelaskan maksud dari tulisan tersebut dan mengartikan per baris[hari ke 2](10)
\end{enumerate}


\section{Penanganan Error}
Dari percobaan yang dilakukan di atas, apabila mendapatkan error maka:

\begin{enumerate}
	\item
	skrinsut error[hari ke 2](10)
	\item
Tuliskan kode eror dan jenis errornya [hari ke 2](10)
	\item
Solusi pemecahan masalah error tersebut[hari ke 2](10)

\end{enumerate}

