\chapter{Membangun Model Prediksi}

Untuk pratikum saati ini menggunakan buku \textit{Python Artificial Intelligence Projects for Beginners}\cite{eckroth2018python}. Dengan praktek menggunakan python 3 dan editor anaconda dan library python scikit-learn.
Dataset ada di https://github.com/PacktPublishing/Python-Artificial-Intelligence-Projects-for-Beginners .
Tujuan pembelajaran pada pertemuan pertama antara lain:
\begin{enumerate}
\item
Mengerti implementasi klasifikasi
\item
Memahami data set, training dan testing data
\item
Memahami Decission tree.
\item
Memahami information gain dan entropi.
\end{enumerate}
Tugas dengan cara dikumpulkan dengan pull request ke github dengan menggunakan latex pada repo yang dibuat oleh asisten riset. Kode program menggunakan input listing ditaruh di folder src ekstensi .py dan dipanggil ke latex dengan input listings. Tulisan dan kode tidak boleh plagiat, menggunakan bahasa indonesia yang sesuai dengan gaya bahasa buku teks.

\section{Teori}
Praktek teori penunjang yang dikerjakan(nilai 5 per nomor, untuk hari pertama) :
\begin{enumerate}
\item
Jelaskan apa itu binary classification dilengkapi ilustrasi gambar sendiri
\item
Jelaskan apa itu supervised learning dan unsupervised learning dan clustering dengan ilustrasi gambar sendiri.
\item
Jelaskan apa itu evaluasi dan akurasi dari buku dan disertai ilustrasi contoh dengan gambar sendiri
\item
Jelaskan bagaimana cara membuat dan membaca confusion matrix, buat confusion matrix buatan sendiri.
\item
Jelaskan bagaimana K-fold cross validation bekerja dengan gambar ilustrasi contoh buatan sendiri.
\item
Jelaskan apa itu decision tree dengan gambar ilustrasi contoh buatan sendiri.
\item
Jelaskan apa itu information gain dan entropi dengan gambar ilustrasi buatan sendiri.
\end{enumerate}

\section{scikit-learn}
Dataset ambil di https://github.com/PacktPublishing/Python-Artificial-Intelligence-Projects-for-Beginners folder Chapter01.
Tugas anda adalah, dataset ganti menggunakan \textbf{student-mat.csv} dan mengganti semua nama variabel dari kode di bawah ini dengan nama-nama makanan (NPM mod 3=0), kota (NPM mod 3=1), buah (NPM mod 3=2), . Jalankan satu per satu kode tersebut di spyder dengan menggunakan textit{Run current cell}. Kemudian Jelaskan dengan menggunakan bahasa yang mudah dimengerti dan bebas plagiat dan wajib skrinsut dari komputer sendiri masing masing nomor di bawah ini(nilai 5 masing masing pada hari kedua).

\begin{enumerate}

\item
\begin{verbatim}
	# load dataset (student mat pakenya)
	import pandas as pd
	d = pd.read_csv('student-mat.csv', sep=';')
	len(d)
\end{verbatim}
\item
\begin{verbatim}
	# generate binary label (pass/fail) based on G1+G2+G3 
	# (test grades, each 0-20 pts); threshold for passing is sum>=30
	d['pass'] = d.apply(lambda row: 1 if (row['G1']+row['G2']+row['G3']) 
											>= 35 else 0, axis=1)
	d = d.drop(['G1', 'G2', 'G3'], axis=1)
	d.head()
\end{verbatim}
\item
\begin{verbatim}
	# use one-hot encoding on categorical columns
	d = pd.get_dummies(d, columns=['sex', 'school', 'address', 
									'famsize', 
									'Pstatus', 'Mjob', 'Fjob', 
	                               'reason', 'guardian', 'schoolsup', 
								   'famsup', 'paid', 'activities',
	                               'nursery', 'higher', 'internet', 
									'romantic'])
	d.head()
\end{verbatim}
\item
\begin{verbatim}
	# shuffle rows
	d = d.sample(frac=1)
	# split training and testing data
	d_train = d[:500]
	d_test = d[500:]

	d_train_att = d_train.drop(['pass'], axis=1)
	d_train_pass = d_train['pass']

	d_test_att = d_test.drop(['pass'], axis=1)
	d_test_pass = d_test['pass']

	d_att = d.drop(['pass'], axis=1)
	d_pass = d['pass']

	# number of passing students in whole dataset:
	import numpy as np
	print("Passing: %d out of %d (%.2f%%)" % (np.sum(d_pass), len(d_pass), 
	       100*float(np.sum(d_pass)) / len(d_pass)))
\end{verbatim}
\item 
\begin{verbatim}
	# fit a decision tree
	from sklearn import tree
	t = tree.DecisionTreeClassifier(criterion="entropy", max_depth=5)
	t = t.fit(d_train_att, d_train_pass)
\end{verbatim}
\item
\begin{verbatim}
	# visualize tree
	import graphviz
	dot_data = tree.export_graphviz(t, out_file=None, label="all", 
									impurity=False, proportion=True,
	                                feature_names=list(d_train_att), 
									class_names=["fail", "pass"], 
	                                filled=True, rounded=True)
	graph = graphviz.Source(dot_data)
	graph
\end{verbatim}
\item
\begin{verbatim}
	# save tree
	tree.export_graphviz(t, out_file="student-performance.dot", 
						 label="all", impurity=False, 
						 proportion=True,
	                     feature_names=list(d_train_att), 
	                     class_names=["fail", "pass"], 
	                     filled=True, rounded=True)
\end{verbatim}
\item
\begin{verbatim}
	t.score(d_test_att, d_test_pass)
\end{verbatim}
\item
\begin{verbatim}
	from sklearn.model_selection import cross_val_score
	scores = cross_val_score(t, d_att, d_pass, cv=5)
	# show average score and +/- two standard deviations away 
	#(covering 95% of scores)
	print("Accuracy: %0.2f (+/- %0.2f)" % (scores.mean(), scores.std() * 2))
\end{verbatim}
\item 
\begin{verbatim}
	for max_depth in range(1, 20):
	    t = tree.DecisionTreeClassifier(criterion="entropy", 
			max_depth=max_depth)
	    scores = cross_val_score(t, d_att, d_pass, cv=5)
	    print("Max depth: %d, Accuracy: %0.2f (+/- %0.2f)" % 
				(max_depth, scores.mean(), scores.std() * 2)
			 )
\end{verbatim}
\item
\begin{verbatim}
	depth_acc = np.empty((19,3), float)
	i = 0
	for max_depth in range(1, 20):
	    t = tree.DecisionTreeClassifier(criterion="entropy", 
			max_depth=max_depth)
	    scores = cross_val_score(t, d_att, d_pass, cv=5)
	    depth_acc[i,0] = max_depth
	    depth_acc[i,1] = scores.mean()
	    depth_acc[i,2] = scores.std() * 2
	    i += 1

	depth_acc
\end{verbatim}
\item 
\begin{verbatim}
	import matplotlib.pyplot as plt
	fig, ax = plt.subplots()
	ax.errorbar(depth_acc[:,0], depth_acc[:,1], yerr=depth_acc[:,2])
	plt.show()
\end{verbatim}

\end{enumerate}


\section{Penanganan Error}
Dari percobaan yang dilakukan di atas, error yang kita dapatkan di dokumentasikan dan di selesaikan(nilai 5 hari kedua):

\begin{enumerate}
	\item
skrinsut error
	\item
Tuliskan kode eror dan jenis errornya
	\item
Solusi pemecahan masalah error tersebut

\end{enumerate}

